\documentclass[11pt,preprint, authoryear]{elsarticle}

\usepackage{lmodern}
%%%% My spacing
\usepackage{setspace}
\setstretch{1.2}
\DeclareMathSizes{12}{14}{10}{10}

% Wrap around which gives all figures included the [H] command, or places it "here". This can be tedious to code in Rmarkdown.
\usepackage{float}
\let\origfigure\figure
\let\endorigfigure\endfigure
\renewenvironment{figure}[1][2] {
    \expandafter\origfigure\expandafter[H]
} {
    \endorigfigure
}

\let\origtable\table
\let\endorigtable\endtable
\renewenvironment{table}[1][2] {
    \expandafter\origtable\expandafter[H]
} {
    \endorigtable
}


\usepackage{ifxetex,ifluatex}
\usepackage{fixltx2e} % provides \textsubscript
\ifnum 0\ifxetex 1\fi\ifluatex 1\fi=0 % if pdftex
  \usepackage[T1]{fontenc}
  \usepackage[utf8]{inputenc}
\else % if luatex or xelatex
  \ifxetex
    \usepackage{mathspec}
    \usepackage{xltxtra,xunicode}
  \else
    \usepackage{fontspec}
  \fi
  \defaultfontfeatures{Mapping=tex-text,Scale=MatchLowercase}
  \newcommand{\euro}{€}
\fi

\usepackage{amssymb, amsmath, amsthm, amsfonts}

\def\bibsection{\section*{References}} %%% Make "References" appear before bibliography


\usepackage[round]{natbib}
\bibliographystyle{plainnat}

\usepackage{longtable}
\usepackage[margin=2.3cm,bottom=2cm,top=2.5cm, includefoot]{geometry}
\usepackage{fancyhdr}
\usepackage[bottom, hang, flushmargin]{footmisc}
\usepackage{graphicx}
\numberwithin{equation}{section}
\numberwithin{figure}{section}
\numberwithin{table}{section}
\setlength{\parindent}{0cm}
\setlength{\parskip}{1.3ex plus 0.5ex minus 0.3ex}
\usepackage{textcomp}
\renewcommand{\headrulewidth}{0.2pt}
\renewcommand{\footrulewidth}{0.3pt}

\usepackage{array}
\newcolumntype{x}[1]{>{\centering\arraybackslash\hspace{0pt}}p{#1}}

%%%%  Remove the "preprint submitted to" part. Don't worry about this either, it just looks better without it:
\makeatletter
\def\ps@pprintTitle{%
  \let\@oddhead\@empty
  \let\@evenhead\@empty
  \let\@oddfoot\@empty
  \let\@evenfoot\@oddfoot
}
\makeatother

 \def\tightlist{} % This allows for subbullets!

\usepackage{hyperref}
\hypersetup{breaklinks=true,
            bookmarks=true,
            colorlinks=true,
            citecolor=blue,
            urlcolor=blue,
            linkcolor=blue,
            pdfborder={0 0 0}}


% The following packages allow huxtable to work:
\usepackage{siunitx}
\usepackage{multirow}
\usepackage{hhline}
\usepackage{calc}
\usepackage{tabularx}
\usepackage{booktabs}
\usepackage{caption}
\usepackage{colortbl}

\urlstyle{same}  % don't use monospace font for urls
\setlength{\parindent}{0pt}
\setlength{\parskip}{6pt plus 2pt minus 1pt}
\setlength{\emergencystretch}{3em}  % prevent overfull lines
\setcounter{secnumdepth}{5}

%%% Use protect on footnotes to avoid problems with footnotes in titles
\let\rmarkdownfootnote\footnote%
\def\footnote{\protect\rmarkdownfootnote}
\IfFileExists{upquote.sty}{\usepackage{upquote}}{}

%%% Include extra packages specified by user
% Insert custom packages here as follows
% \usepackage{tikz}

%%% Hard setting column skips for reports - this ensures greater consistency and control over the length settings in the document.
%% page layout
%% paragraphs
\setlength{\baselineskip}{12pt plus 0pt minus 0pt}
\setlength{\parskip}{12pt plus 0pt minus 0pt}
\setlength{\parindent}{0pt plus 0pt minus 0pt}
%% floats
\setlength{\floatsep}{12pt plus 0 pt minus 0pt}
\setlength{\textfloatsep}{20pt plus 0pt minus 0pt}
\setlength{\intextsep}{14pt plus 0pt minus 0pt}
\setlength{\dbltextfloatsep}{20pt plus 0pt minus 0pt}
\setlength{\dblfloatsep}{14pt plus 0pt minus 0pt}
%% maths
\setlength{\abovedisplayskip}{12pt plus 0pt minus 0pt}
\setlength{\belowdisplayskip}{12pt plus 0pt minus 0pt}
%% lists
\setlength{\topsep}{10pt plus 0pt minus 0pt}
\setlength{\partopsep}{3pt plus 0pt minus 0pt}
\setlength{\itemsep}{5pt plus 0pt minus 0pt}
\setlength{\labelsep}{8mm plus 0mm minus 0mm}
\setlength{\parsep}{\the\parskip}
\setlength{\listparindent}{\the\parindent}
%% verbatim
\setlength{\fboxsep}{5pt plus 0pt minus 0pt}



\begin{document}

\begin{frontmatter}  %

\title{A Descriptive Narrative of Economic Policy Uncertainty in South Africa}

% Set to FALSE if wanting to remove title (for submission)




\author[Add1]{Roan Minnie}
\ead{roanminnie@gmail.com}

\author[Add2]{Nicolaas Johannes Odendaal}
\ead{hanjo.oden@gmail.com}




\address[Add1]{Stellenbosch University, Stellenbosch, South Africa}
\address[Add2]{Stellenbosch University, Stellenbosch, South Africa; Bureau of Economic
Research, South Africa}

\cortext[cor]{Corresponding author: Roan Minnie}


\vspace{1cm}

\begin{keyword}
\footnotesize{
 \\ \vspace{0.3cm}
\textit{JEL classification} 
}
\end{keyword}
\vspace{0.5cm}
\end{frontmatter}



%________________________
% Header and Footers
%%%%%%%%%%%%%%%%%%%%%%%%%%%%%%%%%
\pagestyle{fancy}
\chead{}
\rhead{}
\lfoot{}
\rfoot{\footnotesize Page \thepage\\}
\lhead{}
%\rfoot{\footnotesize Page \thepage\ } % "e.g. Page 2"
\cfoot{}

%\setlength\headheight{30pt}
%%%%%%%%%%%%%%%%%%%%%%%%%%%%%%%%%
%________________________

\headsep 35pt % So that header does not go over title




\section{\texorpdfstring{Introduction
\label{sec_intro}}{Introduction }}\label{introduction}

The aim of this paper is to provide a descriptive narrative of economic
policy uncertainty in South Africa since \ldots{} To this end, an EPU
index is constructed using text data from online news sources.

\section{\texorpdfstring{Economic policy uncertainty in existing
literature
\label{sec_litreview}}{Economic policy uncertainty in existing literature }}\label{economic-policy-uncertainty-in-existing-literature}

Existing literature provide several definitions of uncertainty. In
general terms, Jurado, Ludvigson, and Ng
(\protect\hyperlink{ref-Jurado2015}{2015}) define uncertainty as
volatility in shocks that cannot be forecasted by economic agents. Bloom
(\protect\hyperlink{ref-Bloom2014}{2014}) echo this by describing
uncertainty as a concept that occupies the minds of agents in relation
to possible future events. More specifically, Baker, Bloom, and Davis
(\protect\hyperlink{ref-Baker2016}{2016}) define economic policy
uncertainty not only as uncertainty about policies as such, but also
about who is making policy decisions and the effects thereof. It is
important to note that uncertainty is defined in terms of all agents in
the economy - as stated by Bloom
(\protect\hyperlink{ref-Bloom2014}{2014}) - consumers, managers and
policymakers. The remainder of this paper will thus refer to EPU at this
aggregated level.

The sole purpose of this paper is to provide a narrative of the
evolution of economic policy uncertainty in South Africa that is heavily
supported by descriptive analysis. An obvious extension to this research
is to investigate the impact of uncertainty on key variables of
interest. Albeit not within the scope of this paper, it is still
pertinent to motivate the rationale for constructing a measure of EPU by
looking at the relationships existing literature have uncovered. A
delineation between a macro- and microeconomic focus is immediate in
this literature. This paper will suffice with motivating the usefulness
of a measure for EPU by referring to this literature below.

In terms of the microeconomy, Bachmann, Elstner, and Sims
(\protect\hyperlink{ref-Bachmann2013}{2013}) find that there is a
decline in production and employment in response to a shock in EPU.
Bloom (\protect\hyperlink{ref-Bloom2014}{2014}) echoes that the level of
uncertainty at a microeconomic level (i.e.~individual industries, firms
and plants) is usually higher during times of economic recessions.
However, not sufficiently controlling for the economic conditions that
can also influence production and employment can confound the estimate
of the impact of EPU on these variables. To this end, Caggiano,
Castelnuovo, and Groshenny (\protect\hyperlink{ref-Caggiano2014}{2014})
show that not taking into account that uncertainty shocks occur mostly
in economic recessions, leads to an underestimation of the negative
relationship between uncertainty and unemployment. It is immediate that
aggregating the industry/firm level effects will feed into the
macroeconomy.

At the macroeconomic level, Baker, Bloom, and Davis
(\protect\hyperlink{ref-Baker2016}{2016}) find that rising uncertainty
in the United States is a leading indicator for declining investment,
production and employment. However, since the global financial crisis
(GFC), most macroeconomic literature pertaining to uncertainty have been
concerned with linkages to financial markets. Caldara et al.
(\protect\hyperlink{ref-Caldara2016}{2016}) go as far in stating that
the interaction between uncertainty and financial shocks is ``toxic''
and that the GFC was likely due to an acute manifestation of this. Other
scholars establish relationships between EPU and the equity option
market (Kelly, Pástor, and Veronesi
\protect\hyperlink{ref-Kelly2016}{2016}), increased stock price
volatility (Baker, Bloom, and Davis
\protect\hyperlink{ref-Baker2016}{2016}) and increased risk premia
(Pástor and Veronesi \protect\hyperlink{ref-Pastor2013}{2013}).

The above discussion illustrates the usefulness of a measure for EPU. It
now remains to be ascertained how exactly to measure it. Earlier
literature (see e.g. Bachmann, Elstner, and Sims
(\protect\hyperlink{ref-Bachmann2013}{2013})) made use of survey data
and constructed an EPU index by measuring the discrepancies between
respondents' answers to forward-looking questions. Redl
(\protect\hyperlink{ref-Redl2015}{2015}) takes an alternative approach,
employing inter alia data from professional forecasting competitions and
scrutinising official economic reviews released by the South African
Reserve Bank to construct an EPU index.

More recently, a novel approach by Baker, Bloom, and Davis
(\protect\hyperlink{ref-Baker2016}{2016}) garnered academic praise and
an increasing number of scholars followed suit. The approach developed
by Baker, Bloom, and Davis (\protect\hyperlink{ref-Baker2016}{2016})
gauge EPU by extracting information from newspaper articles. The most
basic index is constructed using the frequency of articles containing
words related to uncertainty. Refinements on the basic index attempt to,
inter alia, control for linguistic modality (Tobback et al.
\protect\hyperlink{ref-Tobback2018}{2018}), eliminate the need for human
classification of articles by using support vector machines
(Azqueta-Gavaldón \protect\hyperlink{ref-Azqueta-Gavaldon2017}{2017})
and provide a greater level of disaggregation in terms of policy
categories (Baker, Bloom, and Davis
\protect\hyperlink{ref-Baker2016}{2016}). The methodologies of these
papers are not discussed here as it overlaps greatly with the
methodology discussed in section \ref{sec_method}.

\section{\texorpdfstring{Constructing an Economic Policy Uncertainty
Index for South Africa
\label{sec_EPU}}{Constructing an Economic Policy Uncertainty Index for South Africa }}\label{constructing-an-economic-policy-uncertainty-index-for-south-africa}

\subsection{\texorpdfstring{Data \label{sec_data}}{Data }}\label{data}

\subsection{\texorpdfstring{Methodology
\label{sec_method}}{Methodology }}\label{methodology}

This paper presents two sets of indices - the naive and the refined -
both following the same methodology. The delineating factor is the way
in which the uncertainty score for each article is calculated. As such,
this calculation for the respective indices is discussed before
explaining the general methodology of constructing the indices.

For the naive approach, the uncertainty score is simply a binary
variable that is equal to one if the article contains the word
``uncertain'' at least once and zero otherwise. Albeit being able to
quantify uncertainty to an extent, the naive index is rigid in its
application. Tobback et al. (\protect\hyperlink{ref-Tobback2018}{2018})
state that every article matching the aforementioned criteria forms part
of the naive index, regardless of the entity that the uncertainty is
related to. To control for this, the naive index is calculated for
uncertainty pertaining to specific topics. This is achieved by
identifying a subset of the articles.\textless{}-- I will elaborate on
this point once I've indentified the categories that I am constructing
the indices for e.g.~finance, stock markets etc--\textgreater{}

A further shortcoming of the naive index is that articles are weighted
equally, i.e.~all articles containing the word ``uncertain'' contribute
one towards the uncertainty score. In reality, there is a range of words
that can convey uncertainty and not all articles convey the same level
of uncertainty. Therefore, the refined uncertainty index evaluates the
raw text data for words contained in the Loughran and Mcdonald
(\protect\hyperlink{ref-Loughran2016}{2016}) dictionary pertaining to
uncertainty. In doing so, the uncertainty measure is a continuous
variable and each article's contribution to the uncertainty score is
proportional to this score. \textless{}--elaborate more on this method
(frequency over wordcount) as well as the choice of dictionary
--\textgreater{}

Having demonstrated the different ways in which the naive- and refined
approach measure the uncertainty score it can now be shown how these
scores are translated into an index. As mentioned, the methodology to
calculate the indices are the same and the only difference is the way
that uncertainty is measured. As such the general methodology developed
by Baker, Bloom, and Davis (\protect\hyperlink{ref-Baker2016}{2016}) is
explained below.

First, the subset of the data is identified by filtering the data to
only articles containing the relevant word. The daily uncertainty score
is then calculated by adding the scores for all the articles in the
subset. Since the number of articles in the subset varies per day, the
score is standardised by dividing by by the number of articles in the
subset. The variance of the daily standardised score is then calculated
for every month. Finally, the first month of the dataset is chosen as
the reference period and subsequent months are standardised by dividing
by the reference period.

The above explanation can be framed mathematically as follows: Denote an
article as \(a_{jit}\) and define its uncertainty score as \(us_{jit}\)
where \(j\) denotes the article, \(i\) the day and \(t\) the month. The
uncertainty score per article is aggregated to obtain the daily
uncertainty score, \(\sum_{j=1}^J us_{jit} = US_{it}\). The standardised
score is calculated by dividing this by the total articles pertaining to
a certain topic published on a day,
\(X_{it}=\frac{US_{it}}{\sum_{j=1}^J a_{jit}}\). Finally, the variance
is calculated by month and scaled by the first month to obtain the index
value for each month, \(index = \frac{var(X_t)}{var(X_0)}\).

\subsection{\texorpdfstring{Results
\label{sec_results}}{Results }}\label{results}

\section{\texorpdfstring{The evolution of Economic Policy Uncertainty in
South Africa
\label{sec_discuss}}{The evolution of Economic Policy Uncertainty in South Africa }}\label{the-evolution-of-economic-policy-uncertainty-in-south-africa}

\section{\texorpdfstring{Conclusion
\label{sec_conclude}}{Conclusion }}\label{conclusion}

\newpage

\section*{References}\label{references}
\addcontentsline{toc}{section}{References}

\hypertarget{refs}{}
\hypertarget{ref-Azqueta-Gavaldon2017}{}
Azqueta-Gavaldón, Andrés. 2017. ``Developing News-Based Economic Policy
Uncertainty Index with Unsupervised Machine Learning.'' \emph{Economics
Letters} 158 (September): 47--50.
doi:\href{https://doi.org/10.1016/j.econlet.2017.06.032}{10.1016/j.econlet.2017.06.032}.

\hypertarget{ref-Bachmann2013}{}
Bachmann, Rüdiger, Steffen Elstner, and Eric R Sims. 2013. ``Uncertainty
and Economic Activity: Evidence from Business Survey Data.''
\emph{American Economic Journal: Macroeconomics} 5 (2): 217--49.
doi:\href{https://doi.org/10.1257/mac.5.2.217}{10.1257/mac.5.2.217}.

\hypertarget{ref-Baker2016}{}
Baker, Scott R., Nicholas Bloom, and Steven J. Davis. 2016. ``Measuring
Economic Policy Uncertainty.'' \emph{The Quarterly Journal of Economics}
131 (4): 1593--1636.
doi:\href{https://doi.org/10.1093/qje/qjw024}{10.1093/qje/qjw024}.

\hypertarget{ref-Bloom2014}{}
Bloom, Nicholas. 2014. ``Fluctuations in Uncertainty.'' \emph{Journal of
Economic Perspectives} 28 (2): 153--76.
doi:\href{https://doi.org/10.1257/jep.28.2.153}{10.1257/jep.28.2.153}.

\hypertarget{ref-Caggiano2014}{}
Caggiano, Giovanni, Efrem Castelnuovo, and Nicolas Groshenny. 2014.
``Uncertainty Shocks and Unemployment Dynamics in U.S. Recessions.''
\emph{Journal of Monetary Economics} 67 (October): 78--92.
doi:\href{https://doi.org/10.1016/j.jmoneco.2014.07.006}{10.1016/j.jmoneco.2014.07.006}.

\hypertarget{ref-Caldara2016}{}
Caldara, Dario, Cristina Fuentes-Albero, Simon Gilchrist, and Egon
Zakrajšek. 2016. ``The Macroeconomic Impact of Financial and Uncertainty
Shocks.'' \emph{European Economic Review} 88 (September): 185--207.
doi:\href{https://doi.org/10.1016/j.euroecorev.2016.02.020}{10.1016/j.euroecorev.2016.02.020}.

\hypertarget{ref-Jurado2015}{}
Jurado, Kyle, Sydney C. Ludvigson, and Serena Ng. 2015. ``Measuring
Uncertainty.'' \emph{American Economic Review} 105 (3): 1177--1216.
doi:\href{https://doi.org/10.1257/aer.20131193}{10.1257/aer.20131193}.

\hypertarget{ref-Kelly2016}{}
Kelly, Bryan, Ľuboš Pástor, and Pietro Veronesi. 2016. ``The Price of
Political Uncertainty: Theory and Evidence from the Option Market: The
Price of Political Uncertainty.'' \emph{The Journal of Finance} 71 (5):
2417--80.
doi:\href{https://doi.org/10.1111/jofi.12406}{10.1111/jofi.12406}.

\hypertarget{ref-Loughran2016}{}
Loughran, Tim, and Bill Mcdonald. 2016. ``Textual Analysis in Accounting
and Finance: A Survey: TEXTUAL ANALYSIS IN ACCOUNTING AND FINANCE.''
\emph{Journal of Accounting Research} 54 (4): 1187--1230.
doi:\href{https://doi.org/10.1111/1475-679X.12123}{10.1111/1475-679X.12123}.

\hypertarget{ref-Pastor2013}{}
Pástor, Ľuboš, and Pietro Veronesi. 2013. ``Political Uncertainty and
Risk Premia.'' \emph{Journal of Financial Economics} 110 (3): 520--45.
doi:\href{https://doi.org/10.1016/j.jfineco.2013.08.007}{10.1016/j.jfineco.2013.08.007}.

\hypertarget{ref-Redl2015}{}
Redl, Chris. 2015. ``Macroeconomic Uncertainty in South Africa,'' 34.

\hypertarget{ref-Tobback2018}{}
Tobback, Ellen, Hans Naudts, Walter Daelemans, Enric Junqué de Fortuny,
and David Martens. 2018. ``Belgian Economic Policy Uncertainty Index:
Improvement Through Text Mining.'' \emph{International Journal of
Forecasting} 34 (2): 355--65.
doi:\href{https://doi.org/10.1016/j.ijforecast.2016.08.006}{10.1016/j.ijforecast.2016.08.006}.

% Force include bibliography in my chosen format:

\bibliographystyle{Tex/Texevier}
\bibliography{Tex/ref}





\end{document}
